\subsection*{Definiciones}
Sea \( (V, +, \cdot) \) un \( k \)-espacio vectorial finitamente generado y \( X \subseteq V \).
Diremos que \( X \) es base de \( V \) si \( \langle X \rangle = V \) y \( X \) es linealmente independiente, donde \( \langle X \rangle \) denota al \( k \)-subespacio vectorial de \( V \) generado por \( X \).

\paragraph{Nota:}
Base NO se define en espacios que no son finitamente generados (hay una noción análoga pero no se ve en este curso, pues se tiene que tener mucho cuidado con la lógica).

\subsection*{Definición}
Sea \( (V, +, \cdot) \) un \( k \)-espacio vectorial finitamente generado y \( B \) cualquier base de \( V \).

Diremos que la dimensión de \( V \) (como \( k \)-espacio vectorial), \( \dim_k V \), es \( |B| \).

\begin{enumerate}
    \item[3)] Sea \( X \subseteq V \) conjunto linealmente independiente. Entonces \( \exists B \) base de \( V \), \( X \subseteq B \).
    
    \item[\textbf{Corolario.}] Sea \( X \subseteq V \) conjunto linealmente independiente. Entonces \( |X| \leq n \).
    
    \item[4)] Sea \( Y \subseteq V \) conjunto generador de \( V \). Entonces
    \begin{enumerate}
        \item[4.1)] \( \exists \hat{Y} \subseteq Y \) subconjunto finito tal que \( \langle \hat{Y} \rangle = V \).
        \item[4.2)] \( \exists B \) base de \( V \), \( Y \subseteq B \).
    \end{enumerate}
    
    \item[5)] Sea \( n = \dim_k V \). Si \( X \subseteq V \) es linealmente independiente en \( V \) y \( |X| = n \) entonces \( X \) es base de \( V \).
    
    \item[6)] Sea \( n = \dim_k V \). Si \( Y \subseteq V \) genera a \( V \) y \( |Y| = n \) entonces \( Y \) es base de \( V \).
\end{enumerate}


\subsection*{Ejemplos}

\subsection*{Problemas}