\documentclass{article}
\usepackage{amsmath}

\begin{document}

Dada la parábola \(y = x^2\), encontrar:

\begin{enumerate}
\item Una ecuación paramétrica vectorial para la curva \(\vec{r}(t)\).

Para una parábola dada por \(y = x^2\), podemos elegir \(x = t\) como nuestro parámetro. Entonces, la ecuación paramétrica vectorial para la curva es:

\[
\vec{r}(t) = (t, t^2)
\]

\item La longitud de la curva desde \((-2, 4)\) hasta \((1, 1)\).

La longitud de una curva dada por una función \(y = f(x)\) desde \(x = a\) hasta \(x = b\) se calcula utilizando la integral:

\[
L = \int_{a}^{b} \sqrt{1 + [f'(x)]^2} dx
\]

En este caso, \(f'(x) = 2x\), \(a = -2\) y \(b = 1\). Por lo tanto, la longitud de la curva es:

\[
L = \int_{-2}^{1} \sqrt{1 + (2x)^2} dx \approx 6.1257
\]

\item La curvatura \(\kappa\) y la torsión \(\tau\) para todo punto.

Para una curva en el plano, la curvatura \(\kappa\) se calcula como:

\[
\kappa = \frac{|\vec{r}'(t) \times \vec{r}''(t)|}{|\vec{r}'(t)|^3}
\]

Y la torsión \(\tau\) es cero, ya que la curva está en el plano.

Las primeras y segundas derivadas de \(\vec{r}(t)\) son:

\[
\vec{r}'(t) = (1, 2t)
\]
\[
\vec{r}''(t) = (0, 2)
\]

Por lo tanto, la curvatura \(\kappa\) es:

\[
\kappa = \frac{|\vec{r}'(t) \times \vec{r}''(t)|}{|\vec{r}'(t)|^3} = \frac{2}{5 \sqrt{5}}
\]

Por lo tanto, para cada punto en la parábola \(y = x^2\), la curvatura \(\kappa\) es \(\frac{2}{5 \sqrt{5}}\) y la torsión \(\tau\) es 0.

\end{enumerate}

\end{document}
