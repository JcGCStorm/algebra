\documentclass{article}
\usepackage{amsmath, amssymb}

\title{Álgebra Superior II - Tarea 6 - Ejericio 2}
\author{David Rivera Morales}
\date{7 de octubre de 2023}

\begin{document}
\maketitle

\section*{Dé un contraejemplo para atestiguar la falsedad de cada una de las siguientes afirmaciones:}

\section*{Problema 1}
Para todo entero \( n > 1 \) y para cualesquiera \( a, b, c \in \mathbb{Z} \), siempre que \( a \cdot c \equiv b \cdot c \) (mod \( n \)), se tiene que \( a \equiv b \) (mod \( n \)).

\section*{Contraejemplo}
Para encontar encontrar un contraejemplo necesitamos encontrar valores de \( n \), \( a \), \( b \), y \( c \) tales que:

\begin{enumerate}
    \item \( n > 1 \)
    \item \( a \cdot c \equiv b \cdot c \) (mod \( n \))
    \item \( a \not\equiv b \) (mod \( n \))
\end{enumerate}

Consideremos el módulo \( n = 6 \). Tomemos:

\[
\begin{aligned}
    a &= 2 \\
    b &= 4 \\
    c &= 3 \\
\end{aligned}
\]

Aunque \( 2 \times 3 \equiv 4 \times 3 \) (mod 6), tenemos \( 2 \not\equiv 4 \) (mod 6).

Por lo tanto, los valores:
\[
\begin{aligned}
    n &= 6 \\
    a &= 2 \\
    b &= 4 \\
    c &= 3 \\
\end{aligned}
\]
son un contraejemplo válido para refutar la proposición.

\section*{Problema 2}
Sean \( n > 1 \) y \( a, b, c, d \in \mathbb{Z} \). Si \( a+c \equiv b+d \) (mod \( n \)), entonces \( a \equiv b \) (mod \( n \)) y \( c \equiv d \) (mod \( n \)).

\section*{Contraejemplo}

Para encontar un contraejemplo válido deseamos encontrar valores de \( n \), \( a \), \( b \), \( c \), y \( d \) tales que:

\begin{enumerate}
    \item \( n > 1 \)
    \item \( a + c \equiv b + d \) (mod \( n \))
    \item \( a \not\equiv b \) (mod \( n \)) o \( c \not\equiv d \) (mod \( n \))
\end{enumerate}

Intentemos con el módulo \( n = 5 \). Elijamos:
\[
\begin{aligned}
    a &= 1 \\
    b &= 2 \\
    c &= 4 \\
    d &= 3 \\   
\end{aligned}
\]

Observamos que:
\[
\begin{aligned}
    a + c &= 1 + 4 = 5 \\
    b + d &= 2 + 3 = 5 \\
\end{aligned}
\]

Por lo tanto, \( a + c \equiv b + d \) (mod 5).

Sin embargo:
\[
\begin{aligned}
    a &= 1 \not\equiv 2 = b \quad \text{(mod 5)} \\
    c &= 4 \not\equiv 3 = d \quad \text{(mod 5)} \\
\end{aligned}
\]

Así que \( a \) no es congruente con \( b \) y \( c \) no es congruente con \( d \) módulo 5.

Entonces los valores:
\[
\begin{aligned}
    n &= 5 \\
    a &= 1 \\
    b &= 2 \\
    c &= 4 \\
    d &= 3 \\
\end{aligned}
\]
son un contraejemplo válido para refutar la proposición.

\end{document}
