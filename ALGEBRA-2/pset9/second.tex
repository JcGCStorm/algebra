\documentclass{article}
\usepackage{amsmath}
\begin{document}

\title{Búsqueda de Coeficientes para la Divisibilidad de un Polinomio}
\author{}
\date{}
\maketitle

\section*{Problema}
Se busca encontrar los coeficientes \( a, b \in \mathbb{Q} \) tales que el polinomio \( x^3 + ax^2 + bx + 5 \) sea divisible por \( x^2 + x \) en \( \mathbb{Q}[x] \).

\section*{Procedimiento}

Para que \( x^3 + ax^2 + bx + 5 \) sea divisible por \( x^2 + x \), las raíces de \( x^2 + x \), es decir, 0 y -1, deben ser también raíces del primer polinomio.

\begin{enumerate}
    \item \textbf{Aplicación de la Raíz \( x = 0 \):}
    \begin{equation*}
        0^3 + a \cdot 0^2 + b \cdot 0 + 5 = 5
    \end{equation*}
    Esto nos da la ecuación \( 5 = 0 \), que no es posible en \( \mathbb{Q} \).

    \item \textbf{Aplicación de la Raíz \( x = -1 \):}
    \begin{equation*}
        (-1)^3 + a \cdot (-1)^2 + b \cdot (-1) + 5 = a - b - 1 + 5
    \end{equation*}
    Esto nos da la ecuación \( a - b + 4 = 0 \).
\end{enumerate}

\section*{Conclusión}

No hay solución para \( a \) y \( b \) en \( \mathbb{Q} \) que haga que el polinomio \( x^3 + ax^2 + bx + 5 \) sea divisible por \( x^2 + x \), ya que la aplicación de las raíces lleva a un sistema de ecuaciones sin solución en los números racionales.

\end{document}
