\documentclass{article}
\author{David Rivera Morales}
\title{Máximo Común Divisor usando el Algoritmo de División}
\date{10 de septiembre de 2023}


\begin{document}

\maketitle

\section*{Demostración por Inducción}

Para demostrar que el producto de tres números enteros consecutivos es divisible por 6, definamos la proposición \( P(n) \) tal que:
\[ P(n): n(n + 1)(n + 2) \text{ es divisible por 6.} \]

\textbf{Caso base:} Consideremos \( n = 1 \):
\[ P(1): 1(1 + 1)(1 + 2) = 6 \]
que es divisible por 6. Por lo tanto, \( P(n) \) es verdadero para \( n = 1 \).

\textbf{Hipótesis de inducción (HI):} Supongamos que \( P(k) \) es verdadero para algún número natural \( k \), es decir:
\[ k(k + 1)(k + 2) \text{ es divisible por 6.} \]

\textbf{Paso inductivo:} Queremos demostrar que \( P(k + 1) \) es verdadero, suponiendo que \( P(k) \) es verdadero.

Expresando \( (k+1)(k+2)(k+3) \) usando distribución:
\[ (k+1)(k+2)(k+3) = k(k+1)(k+2) + 3(k+1)(k+2) \]

Por hipótesis de inducción, sabemos que \( k(k+1)(k+2) \) es divisible por 6. Adicionalmente, \( (k+1)(k+2) \) es divisible por 6 porque al menos uno de los términos, \( (k+1) \) o \( (k+2) \), es par.

Por lo tanto, \( P(k + 1) \) es verdadero.

Hemos demostrado que si \( P(k) \) es verdadero, entonces \( P(k+1) \) también lo es.

Por el principio de inducción matemática, concluimos que \( n(n+1)(n+2) \) es divisible por 6 para todos los números naturales \( n \).

\end{document}
