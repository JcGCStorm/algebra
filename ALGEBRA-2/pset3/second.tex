\documentclass{article}
\usepackage{amsmath}
\usepackage[margin=0.5in]{geometry} % Reduce margins further to 0.5 inches.

\author{David Rivera Morales}
\title{Máximo Común Divisor usando el Algoritmo de División}
\date{10 de septiembre de 2023}


\begin{document}

\maketitle

\section*{Máximo Común Divisor usando el Algoritmo de División}

\subsection*{Para \(a = 434\) y \(b = 31\)}
\begin{align*}
434 & = 31 \times 14 + 0 &&\text{(Dividimos 434 entre 31, residuo 0)} \\
\therefore \text{MCD}(434,31) & = 31 
\end{align*}

\subsection*{Para \(a = 8611\) y \(b = -17\)}
Dado que el MCD no cambia si uno de los números es negativo, tomamos \(b = 17\).
\begin{align*}
8611 & = 17 \times 506 + 9 &&\text{(Residuo 9)} \\
17 & = 9 \times 1 + 8 &&\text{(Residuo 8)} \\
9 & = 8 \times 1 + 1 &&\text{(Residuo 1)} \\
8 & = 1 \times 8 + 0 &&\text{(Residuo 0)} \\
\therefore \text{MCD}(8611,-17) & = 1 
\end{align*}

\end{document}
