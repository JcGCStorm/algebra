\documentclass{article}
\usepackage{amsmath, amssymb}

\begin{document}

\title{Demostración por Inducción de la Suma de una Serie Geométrica}
\author{David Rivera Morales}
\date{26 de agosto de 2023}

\maketitle

\section*{Demostración por inducción de la suma de una serie geométrica}

Sea la serie geométrica:
\[
\sum_{i=0}^{n} ar^i
\]
queremos demostrar que:
\[
\sum_{i=0}^{n} ar^i = \frac{a(r^{n+1}-1)}{r-1}
\]
para todo \( n \in \mathbb{N} \).

\textbf{Base de inducción:} \( n = 0 \)

Para \( n = 0 \):
\[
\sum_{i=0}^{0} ar^i = a
\]
y
\[
\frac{a(r^{0+1}-1)}{r-1} = a
\]
Por lo tanto, la fórmula se cumple para \( n = 0 \).

\textbf{Hipótesis de inducción:} Supongamos que la fórmula se cumple para \( n = k \):
\[
\sum_{i=0}^{k} ar^i = \frac{a(r^{k+1}-1)}{r-1}
\]

\textbf{Paso de inducción:} Demostraremos que se cumple para \( n = k+1 \):

P.D.:
\[
\sum_{i=0}^{k+1} ar^i = \frac{a(r^{k+2}-1)}{r-1}    
\]

Desarrollo:
\begin{align*}
\sum_{i=0}^{k+1} ar^i &= \sum_{i=0}^{k} ar^i + ar^{k+1} \\
&= \frac{a(r^{k+1}-1)}{r-1} + ar^{k+1} & \text{(por hipótesis de inducción)} \\
&= \frac{a(r^{k+1} - 1) + ar^{k+1}(r-1) + ar^{k+1} - ar^{k+1}(r-1)}{r-1} & \text{(agregando y restando } ar^{k+1}(r-1) \text{)} \\
&= \frac{a(r^{k+1}-1) + ar^{k+2} - ar^{k+1}}{r-1} & \text{(distribuyendo)} \\
&= \frac{a(r^{k+2}-1)}{r-1}
\end{align*}

Por inducción matemática, la fórmula es cierta para todo \( n \in \mathbb{N} \).

\end{document}
