\documentclass{article}
\usepackage{amsmath, amssymb}

\begin{document}

\title{Solución de Inecuaciones en \( \mathbb{Z} \)}
\author{David Rivera Morales}
\date{\today}

\maketitle


\section*{Demostración por inducción de la suma de una serie geométrica}

Dada la serie geométrica:
\[
\sum_{i=0}^{n} ar^i
\]
queremos demostrar que:
\[
\sum_{i=0}^{n} ar^i = \frac{a(r^{n+1}-1)}{r-1}
\]
para todo \( n \in \mathbb{N} \).

\textbf{Base de inducción:} \( n = 0 \)

Para \( n = 0 \):
\[
\sum_{i=0}^{0} ar^i = a
\]
y
\[
\frac{a(r^{0+1}-1)}{r-1} = a
\]
Por lo tanto, la fórmula es cierta para \( n = 0 \).

\textbf{Hipótesis de inducción:} Supongamos que la fórmula es cierta para \( n = k \).

\textbf{Paso de inducción:} Demostraremos que es cierta para \( n = k+1 \).

\begin{align*}
\sum_{i=0}^{k+1} ar^i &= \sum_{i=0}^{k} ar^i + ar^{k+1} \\
&= \frac{a(r^{k+1}-1)}{r-1} + ar^{k+1} \\
&= \frac{a(r^{k+1}-1) + a(r^{k+1})(r-1)}{r-1} \\
&= \frac{a(r^{k+2}-1)}{r-1}
\end{align*}

Por inducción matemática, la fórmula es cierta para todo \( n \in \mathbb{N} \).

\end{document}
