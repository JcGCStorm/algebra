\documentclass[12pt]{article}
\usepackage{amsmath, amssymb}

\begin{document}

\title{Solución de Inecuaciones en \( \mathbb{Z} \)}
\author{David Rivera Morales}
\date{26 de agosto de 2023}

\maketitle

\section*{Problema}
Dada la inecuación:
\[
2 - x < 6 - 2x < 4 - x
\]
Encuentra todas las \( x \in \mathbb{Z} \) que la satisfacen, utilizando y mencionando en cada paso las propie-
dades de orden de \( \mathbb{Z} \).

\section*{Solución}
Dividimos la inecuación en dos partes utilizando la propiedad transitiva del orden en \( \mathbb{Z} \): \\
1) \(2 - x < 6 - 2x\)
2) \(6 - 2x < 4 - x\)

Resolviendo la primera:
\begin{align*}
2 - x &< 6 - 2x & \text{(Propiedad Aditiva de } \mathbb{Z} \text{: Sumamos } x \text{ a ambos lados)} \\
x &< 4 & \text{(Definición de orden en } \mathbb{Z} \text{)}
\end{align*}

Resolviendo la segunda:
\begin{align*}
6 - 2x &< 4 - x & \text{(Propiedad Aditiva de } \mathbb{Z} \text{: Sumamos } x \text{ y restamos 4 a ambos lados)} \\
x &> 2 & \text{(Definición de orden en } \mathbb{Z} \text{)}
\end{align*}

Juntando ambas soluciones con la propiedad transitiva del orden en \( \mathbb{Z} \):
\[
2 < x < 4
\]
Por lo que \(x = 3\) es la única solución en \(\mathbb{Z}\).

\end{document}
