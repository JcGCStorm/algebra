\documentclass{article}
\usepackage{amsmath, amssymb, amsthm}

\begin{document}

\section*{Demostraciones}

\subsection*{1. Demostración de \(a^2 \equiv 0 \) (mod 4) para \( a \) par}
Si \( a \) es par, entonces puede ser escrito como \( a = 2k \) para algún \( k \in \mathbb{Z} \).

Entonces,
\begin{align*}
a^2 &= (2k)^2 \\
&= 4k^2
\end{align*}
Esto implica que \( a^2 \) es divisible por 4, y por lo tanto, \( a^2 \equiv 0 \) (mod 4).

\subsection*{2. Demostración de \(b^2 \equiv 1 \) (mod 4) para \( b \) impar}
Si \( b \) es impar, entonces puede ser escrito como \( b = 2k + 1 \) para algún \( k \in \mathbb{Z} \).

Entonces,
\begin{align*}
b^2 &= (2k + 1)^2 \\
&= 4k^2 + 4k + 1
\end{align*}
El término \( 4k^2 + 4k \) es claramente divisible por 4, lo que nos deja con \( b^2 \equiv 1 \) (mod 4).

\subsection*{3. Demostración de \(a^2 - b^2 \equiv -1 \) (mod 4)}
Usando las demostraciones anteriores, sabemos que:
\[ a^2 \equiv 0 \] (mod 4) \\
\[ b^2 \equiv 1 \] (mod 4)

Restando las dos congruencias obtenemos:
\[ a^2 - b^2 \equiv 0 - 1 \] (mod 4)
\[ \implies a^2 - b^2 \equiv -1 \] (mod 4).

\end{document}
