\documentclass[12pt,a4paper]{article}
\usepackage{amsmath, amssymb}
\usepackage{geometry}
\usepackage{titlesec}
\usepackage{titling}
\usepackage{parskip}
\usepackage{graphicx}
\geometry{a4paper, margin=1in}

\titleformat{\section}
{\normalfont\Large\bfseries}
{\thesection}{1em}{}

\pretitle{\begin{center}\Huge\bfseries}
\posttitle{\end{center}\vskip 0.5em}
\preauthor{\begin{center}\large\ttfamily}
\postauthor{\end{center}}
\predate{\begin{center}\large}
\postdate{\end{center}}

\begin{document}
\title{Ejercicio 2}
\author{David Rivera Morales}
\date{31 de Agosto de 2023}
\maketitle

\section*{Proposición}
Si \(a, b, n \in \mathbb{N}\) y \(a \leq b\), entonces se cumple que:
\[a^n \leq b^n\]
para todo \(n \in \mathbb{N}\).

\section*{Demostración por Inducción}

\textbf{Base de inducción:} \(n = 1\)

Si \(a \leq b\), entonces \(a^1 \leq b^1\), lo que implica \(a \leq b\). Esto es trivialmente cierto porque se nos da que \(a \leq b\).

\textbf{Paso inductivo:} Asumamos que la proposición es verdadera para \(n = k\). Es decir, asumamos que:
\[a^k \leq b^k\] (Hipótesis inductiva)

Necesitamos demostrar que:
\[a^{k+1} \leq b^{k+1}\]

Dado que \(a \leq b\), multiplicamos ambos lados de la hipótesis inductiva por \(a\):
\[a \cdot a^k \leq a \cdot b^k\]

Dado que \(a \leq b\), multiplicamos ambos lados de la hipótesis inductiva por \(b\):
\[a^k \cdot b \leq b^{k+1}\]

Combinando las dos inecuaciones:
\[a^{k+1} \leq a \cdot b^k \leq b^{k+1}\]

Por lo tanto, \(a^{k+1} \leq b^{k+1}\).

Hemos demostrado que si la proposición es verdadera para \(n = k\), entonces también es verdadera para \(n = k + 1\).

Por el principio de inducción matemática, la proposición \(a^n \leq b^n\) es verdadera para todo \(n \in \mathbb{N}\).

\section*{Problema 2.b}

\textbf{Objetivo:} Encontrar tres enteros \(a, b, c\) tales que \(a | bc\) pero \(a\) no divida a \(b\) y \(a\) no divida a \(c\).

\textbf{Solución:}

Considerando los factores primos, si \(a\) tiene dos factores primos distintos, \(b\) podría tener uno de esos factores y \(c\) el otro. Así, \(b \times c\) tendría ambos factores primos, siendo divisible por \(a\). Pero individualmente, ni \(b\) ni \(c\) serían divisibles por \(a\).

Tomando \(a = 6\) (donde \(6 = 2 \times 3\) y ambos son primos), y eligiendo \(b = 2\) y \(c = 3\), obtenemos que \(a | bc\), pero \(a\) no divide a \(b\) ni a \(c\).

Por lo tanto, \(a = 6\), \(b = 2\), y \(c = 3\) son los enteros que cumplen con las condiciones dadas.

\end{document}
