\documentclass{article}
\usepackage{amsmath, amssymb}

\begin{document}
\title{Ejercicio 1}
\author{David Rivera Morales}
\date{31 de Agosto de 2023}
\maketitle

\section*{1.a Demostración de que $15 | 2^{4n} - 1$ para todo $n \in \mathbb{N}$}

Para probar que $15$ divide a $2^{4n} - 1$, necesitamos demostrar que:
\begin{enumerate}
    \item $3 \mid 2^{4n} - 1$
    \item $5 \mid 2^{4n} - 1$
\end{enumerate}

Si ambas condiciones son verdaderas, entonces $15$ (que es $3 \times 5$) divide a $2^{4n} - 1$.

\section*{1.Prueba de que $3 \mid 2^{4n} - 1$}

Consideremos la secuencia $2^n \mod 3$:
\begin{align*}
2^0 &\equiv 1 \mod 3 \\
2^1 &\equiv 2 \mod 3 \\
2^2 &\equiv 4 \equiv 1 \mod 3 \\
2^3 &\equiv 8 \equiv 2 \mod 3 \\
2^4 &\equiv 16 \equiv 1 \mod 3 \\
\end{align*}

A partir de aquí, podemos observar un patrón: $2^{4n} \equiv 1 \mod 3$ para todo $n \in \mathbb{N}$.

Por lo tanto, $2^{4n} - 1 \equiv 0 \mod 3$, lo que implica que $3 \mid 2^{4n} - 1$.

\section*{2.Prueba de que $5 \mid 2^{4n} - 1$}

Consideremos la secuencia $2^n \mod 5$:
\begin{align*}
2^0 &\equiv 1 \mod 5 \\
2^1 &\equiv 2 \mod 5 \\
2^2 &\equiv 4 \mod 5 \\
2^3 &\equiv 8 \equiv 3 \mod 5 \\
2^4 &\equiv 16 \equiv 1 \mod 5 \\
\end{align*}

Nuevamente, observamos un patrón: $2^{4n} \equiv 1 \mod 5$ para todo $n \in \mathbb{N}$.

Por lo tanto, $2^{4n} - 1 \equiv 0 \mod 5$, lo que implica que $5 \mid 2^{4n} - 1$.

Dado que ambas condiciones (1 y 2) son verdaderas, concluimos que $15 \mid 2^{4n} - 1$ para todo $n \in \mathbb{N}$.

\section*{1.b Indica si la siguiente proposición es verdadera o falsa: "14 divide a 997157"}    

Para determinar si $14$ divide a $997157$, debemos verificar si $997157$ es divisible por $2$ y $7$. 

1. Un número es divisible por $2$ si es par. $997157$ no es divisible por $2$ ya que es impar.

2. Para verificar la divisibilidad por $7$, utilizamos la regla del $7$. Esta regla establece que se toma el último dígito, se duplica y se resta del número original truncado. Si el resultado es $0$ o divisible por $7$, entonces el número original es divisible por $7$. 

Aplicando la regla del $7$ a $997157$:
\begin{align*}
99715 - 2(7) &= 99701 \\
9970 - 2(1) &= 9968 \\
996 - 2(8) &= 980
\end{align*}

El número $980$ es divisible por $7$. Sin embargo, dado que $997157$ no es divisible por $2$, podemos concluir que $14$ no divide a $997157$.

Por lo tanto, la proposición "14 divide a 997157" es falsa.

\end{document}
