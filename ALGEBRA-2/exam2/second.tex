\documentclass{article}
\usepackage{amsmath}

\begin{document}

\section*{Problema de la Póliza de Seguro y su Resolución}

Partimos de la ecuación:
\[ 100c + p - 23 = 200p + 2c \]
Que se simplifica a:
\[ 98c - 199p = 23 \]

Aislando \( c \) obtenemos:
\[ c = \frac{199p + 23}{98} \]

Para encontrar una solución, debemos probar valores enteros para \( p \) hasta que el numerador \( 199p + 23 \) sea divisible por 98, resultando en un valor entero para \( c \) entre 0 y 99.

Probando con \( p = 1 \):
\[ c = \frac{199(1) + 23}{98} = \frac{222}{98} \]
Esto no resulta en un valor entero para \( c \), por lo que \( p = 1 \) no es una solución.

El proceso se repite para diferentes valores de \( p \) hasta encontrar una solución adecuada.

\end{document}
