\documentclass[12pt]{article}

\usepackage{amsmath,amsfonts,amssymb}
\usepackage[utf8]{inputenc}
\usepackage[spanish]{babel}

\title{Examen 4 - Ejericio 1}
\author{David Rivera Morales}
\date{31 de Octubre de 2023}

\begin{document}

\maketitle

\section*{Problema}
Encuentra las raíces cúbicas \(w_k\) de \(i\) y prueba que \(iw_k\) son las raíces cúbicas de 1.

\section*{Solución}
Para encontrar las raíces cúbicas de \(i\), primero representamos \(i\) en su forma polar:
\[ i = e^{i \pi/2} \]

Buscamos un número \(w_k\) tal que \( (w_k)^3 = i \). En forma polar, esto se da como:
\[ (w_k)^3 = e^{i \pi/2} \]
\[ w_k = e^{i \pi/6 + i 2k\pi/3} \]

Donde \( k = 0, 1, 2 \). Por lo tanto, las raíces cúbicas de \(i\) son:
\[ w_0 = e^{i \pi/6} \]
\[ w_1 = e^{i 5\pi/6} \]
\[ w_2 = e^{i 3\pi/2} \]

Multiplicando cada raíz por \(i\), obtenemos:
\[ iw_0 = e^{i 2\pi/3} \]
\[ iw_1 = e^{i 4\pi/3} \]
\[ iw_2 = 1 \]

Por lo que \(iw_k\) son efectivamente las raíces cúbicas de 1.

\end{document}
