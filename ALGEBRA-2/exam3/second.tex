\documentclass[12pt,a4paper]{article}
\usepackage{amsmath}
\usepackage{amsfonts}
\usepackage{amssymb}
\usepackage{amsthm}
\usepackage[utf8]{inputenc}
\usepackage[left=2cm,right=2cm,top=2cm,bottom=2cm]{geometry}

\begin{document}

\title{Ejercicio 2 - Demostración}
\author{David Rivera Morales}
\date{12 de Octubre de 2023}

\maketitle

\section{Pequeño Teorema de Fermat}
El Pequeño Teorema de Fermat establece que si \( p \) es un número primo y \( a \) es un número entero que no es divisible por \(p\), entonces:
\[a^{p-1} \equiv 1 \pmod{p}\]
En palabras menos formales, al elevar un número \( a \) a la potencia \( p-1 \) y tomar el módulo \( p \), obtenemos 1.

\section{Enunciado}
Dado un número primo impar \(p\), demostrar que:
\[1^p + 2^p + \dots + (p-1)^p \equiv 0 \pmod{p}\]

\section{Demostración}
\begin{proof}
Para cada número \(a\) en el conjunto \(\{1, 2, \dots, p-1\}\), existe un único número \(b\) en el mismo conjunto tal que:
\[a \cdot b \equiv 1 \pmod{p}\]
Esto se debe a la existencia de un inverso multiplicativo para cada número \(a\) en el conjunto, módulo \(p\).

Por el Pequeño Teorema de Fermat, tenemos:
\[a^{p-1} \equiv 1 \pmod{p}\]
Multiplicando ambos lados por \(a\), obtenemos:
\[a^p \equiv a \pmod{p}\]

Sumando estos valores para \(a = 1, 2, \dots, p-1\), tenemos:
\[1^p + 2^p + \dots + (p-1)^p \equiv 1 + 2 + \dots + (p-1) \pmod{p}\]

La suma de los primeros \(p-1\) números naturales es:
\[\frac{(p-1)(p)}{2}\]

Dado que \(p\) es impar, \(p-1\) es par. Por lo tanto, \(\frac{p(p-1)}{2}\) es un entero. Tomando módulo \(p\), obtenemos:
\[\frac{p(p-1)}{2} \equiv 0 \pmod{p}\]

Por lo tanto, 
\[1^p + 2^p + \dots + (p-1)^p \equiv 0 \pmod{p}\]
\end{proof}

\end{document}
