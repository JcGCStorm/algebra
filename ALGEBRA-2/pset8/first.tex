\documentclass[12pt]{article}
\usepackage{amsmath, amssymb}

\title{Ejercicio 1 - Tarea 8}
\author{David Rivera Morales}
\date{29 de Octubre de 2023}

\begin{document}

\maketitle

\noindent \textbf{Problema:}

Si \( z = \cos(\theta) + i\sin(\theta) \), expresa:
\begin{enumerate}
    \item \(\frac{1}{1 + z}\)
    \item \(\frac{1}{1 - z}\)
\end{enumerate}
en su forma \(a + bi\).

\vspace{1em}

\noindent \textbf{Solución:}

\begin{enumerate}
    \item Para \(\frac{1}{1 + z}\):
    
    Para encontrar el conjugado de \(1 + z\), se cambia el signo de la parte imaginaria. El conjugado es \(1 - \cos(\theta) - i\sin(\theta)\). 
    
    Multiplicamos el numerador y el denominador por el conjugado para eliminar la parte imaginaria del denominador:

    \[
    \frac{1}{1 + \cos(\theta) + i\sin(\theta)} \times \frac{1 - \cos(\theta) - i\sin(\theta)}{1 - \cos(\theta) - i\sin(\theta)} = 1 - \cos(\theta) - i\sin(\theta)
    \]

    \item Para \(\frac{1}{1 - z}\):

    Para encontrar el conjugado de \(1 - z\), nuevamente cambiamos el signo de la parte imaginaria. El conjugado es \(1 + \cos(\theta) + i\sin(\theta)\).

    Multiplicamos el numerador y el denominador por el conjugado:

    \[
    \frac{1}{1 - \cos(\theta) - i\sin(\theta)} \times \frac{1 + \cos(\theta) + i\sin(\theta)}{1 + \cos(\theta) + i\sin(\theta)} = 1 + \cos(\theta) + i\sin(\theta)
    \]
\end{enumerate}

\end{document}
