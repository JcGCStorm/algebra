\documentclass{article}
\usepackage{amsmath, amssymb}

\title{Ejercicio 2 - Tarea 8}
\author{David Rivera Morales}
\date{29 de Octubre de 2023}

\begin{document}

\maketitle

\section*{Problema}

Usando la expresión de la Fórmula de De Moivre, encuentre las raíces cuartas de \(-16\).

\section*{Solución}

La fórmula de De Moivre nos dice que si \( z = r(\cos \theta + i \sin \theta) \), entonces:

\[
z^n = r^n (\cos(n\theta) + i \sin(n\theta))
\]

Para \(-16\), podemos representarlo en su forma polar como:

\[
-16 = 16(\cos \pi + i \sin \pi)
\]

Buscamos \( z \) tal que:

\[
z^4 = 16(\cos \pi + i \sin \pi)
\]

Usando la fórmula de De Moivre:

\[
z = 16^{\frac{1}{4}} \left( \cos\left(\frac{\pi + 2\pi k}{4}\right) + i \sin\left(\frac{\pi + 2\pi k}{4}\right) \right)
\]

Donde \( k = 0, 1, 2, 3 \) para obtener las 4 raíces cuartas. Los resultados son:

\begin{enumerate}
    \item \( \sqrt{2} + \sqrt{2}i \)
    \item \( -\sqrt{2} + \sqrt{2}i \)
    \item \( -\sqrt{2} - \sqrt{2}i \)
    \item \( \sqrt{2} - \sqrt{2}i \)
\end{enumerate}

\end{document}