\documentclass{article}
\usepackage[spanish]{babel}
\usepackage[utf8]{inputenc}
\usepackage{amsmath, amssymb}

\begin{document}

\textbf{Problema:}

Definimos las operaciones de \( A = \mathbb{Z} \times \mathbb{Z} \) como sigue:

\[
(a_1, b_1) + (a_2, b_2) = (a_1 + a_2, b_1 + b_2)
\]
\[
(a_1, b_1) \cdot (a_2, b_2) = (a_1 \cdot a_2, b_1 \cdot b_2)
\]

Se puede validar que \( A \), con estas operaciones, es un anillo. ¿\( A \) es un anillo abeliano? Si no lo es, muestra un contraejemplo, si sí lo es, haz una demostración feliz.

\textbf{Solución:}

Para probar si \( A \) es abeliano, necesitamos verificar si la suma es conmutativa. Tomemos dos elementos arbitrarios \( (a_1, b_1) \) y \( (a_2, b_2) \) de \( A \) y comprobemos si \( (a_1, b_1) + (a_2, b_2) = (a_2, b_2) + (a_1, b_1) \).

Dadas las operaciones:
\begin{align*}
(a_1, b_1) + (a_2, b_2) &= (a_1 + a_2, b_1 + b_2) \\
(a_1, b_1) \cdot (a_2, b_2) &= (a_1 \cdot a_2, b_1 \cdot b_2)
\end{align*}

Calculamos:
\begin{align*}
(a_1, b_1) + (a_2, b_2) &= (a_1 + a_2, b_1 + b_2) \\
(a_2, b_2) + (a_1, b_1) &= (a_1 + a_2, b_1 + b_2)
\end{align*}

Ambas sumas son iguales, lo que significa que la operación de suma es conmutativa en \( A \).

Por lo tanto, el anillo \( A \) con las operaciones definidas es un anillo abeliano.


\end{document}
