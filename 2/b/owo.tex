\documentclass{article}
\usepackage{amsmath}
\usepackage{amsfonts}

\begin{document}

\title{Demostración}
\maketitle

\section*{Ejercicio 1}

Sean \(x, y, z \in \mathbb{Z}\). Muestra que si \(x < y\) entonces \(x + z < y + z\).

\begin{proof}
\begin{align*}
\text{Dado que } & x < y, \\
& \text{podemos concluir que } y - x \text{ es un número positivo. Esto se debe a que, si } x \text{ es menor que } y, \\
& \text{entonces la diferencia } y - x \text{ será un valor positivo.} \\
& \text{Denotemos esta diferencia como } w, \text{ donde } w > 0. \\
\Rightarrow & y = x + w \\
\Rightarrow & y + z = x + w + z \\
\text{Dado que } & w \text{ es positivo, } w + z \text{ es simplemente } z \text{ desplazado } w \text{ unidades en la dirección positiva.} \\
\text{Así que } & w + z \text{ sigue siendo mayor que } z. \\
\text{Por lo tanto, } & x + z \text{ es menor que } x + w + z, \\
\Rightarrow & x + z < x + w + z \\
\Rightarrow & x + z < y + z \\
\end{align*}
Por lo tanto, hemos demostrado que si \(x < y\) entonces \(x + z < y + z\) para cualesquiera \(x, y, z\) en \(\mathbb{Z}\).
\end{proof}

\end{document}
