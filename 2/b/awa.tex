\documentclass{article}
\usepackage{amsmath}
\usepackage{amsfonts}
\usepackage[spanish]{babel}  % Use the Spanish language option for babel

\begin{document}

\title{Solución a Desigualdades}
\title{Solución a Desigualdades}
\author{David Rivera Morales}
\maketitle

\section*{Ejercicio 2}

Encuentra el conjunto solución en \( \mathbb{Z} \) para las siguientes desigualdades:

\begin{enumerate}
    \item \( 2x < 5 \)
    
    Primero resolvemos la desigualdad:
    \begin{align*}
        2x &< 5 \\
        x &< \frac{5}{2} \\
    \end{align*}
    Dado que \( x \in \mathbb{Z} \), el conjunto solución es 
    \[
    \{ x \in \mathbb{Z} \mid x < \frac{5}{2} \}.
    \]
    Debido a que \( \frac{5}{2} = 2.5 \), y \( x \) debe ser un número entero, los números enteros que satisfacen la desigualdad son aquellos menores que 2.5, es decir, \( \dots, -2, -1, 0, 1, 2 \).
    
    \item \( 3x + 7 < 8 \)
    
    Resolvemos la desigualdad:
    \begin{align*}
        3x + 7 &< 8 \\
        3x &< 1 \\
        x &< \frac{1}{3} \\
    \end{align*}
    Dado que \( x \in \mathbb{Z} \), el conjunto solución es 
    \[
    \{ x \in \mathbb{Z} \mid x < \frac{1}{3} \}.
    \]
    \( \frac{1}{3} \) es aproximadamente 0.333. Dado que \( x \) debe ser un número entero, los números enteros que satisfacen esta desigualdad son aquellos menores que 0.333, es decir, \( \dots, -3, -2, -1, 0 \).
\end{enumerate}

\end{document}
