\documentclass{article}
\usepackage{amsmath}
\usepackage{tikz}

\begin{document}

\textbf{Problema:}
En una reunión clandestina, cierto número de mafiosos se reunió. Durante la reunión, uno de los contadores fue detenido y ofreció información. Los investigadores dedujeron que había 83 personas: cada hombre conocía exactamente a nueve mujeres y cada mujer a nueve hombres. ¿Es verídica esta información?

\bigskip

Para evaluar la veracidad, desglosemos la información:

Denotemos el número de hombres como \( h \) y el de mujeres como \( m \).

Sabemos que:
1. \( h + m = 83 \)

Y, basándonos en las relaciones:

2. Cada hombre tiene conexiones con 9 mujeres, lo que da \( 9h \) conexiones desde hombres a mujeres.
3. De manera similar, hay \( 9m \) conexiones desde mujeres a hombres.

Como estas conexiones son bidireccionales, tenemos:

\[ 9h = 9m \]
\[ \implies h = m \]

Con la primera ecuación:
\[ 2h = 83 \]

Este resultado es inconsistente, pues 83 no es divisible entre 2. Así, la información no puede ser correcta.


Este problema puede verse como una gráfica bipartita.

\begin{center}
\begin{tikzpicture}[scale=1.5]
    % Nodes for men
    \foreach \i in {1,...,5} {
        \node at (0,\i) [circle, draw, fill=blue!30, minimum size=0.8cm] {H};
    }
    % Nodes for women
    \foreach \i in {1,...,5} {
        \node at (4,\i) [circle, draw, fill=red!30, minimum size=0.8cm] {M};
    }

    % Incomplete node representation to signify uncertainty
    \node at (0,6.5) [circle, draw, pattern=north east lines, minimum size=0.8cm] {?};
    \node at (4,6.5) [circle, draw, pattern=north east lines, minimum size=0.8cm] {?};

    % Labeling for the inconsistent part
    \node at (0,7.5) [above] {Hombres (h)};
    \node at (4,7.5) [above] {Mujeres (m)};
    \draw [red,thick,dashed] (0,0.5) -- (0,7) -- (4,7) -- (4,0.5);
    \node at (2,-1) {\textcolor{red}{Cada H conoce 9 M's y viceversa, pero \(h + m = 83\)}};

    % Conexiones (representative)
    \foreach \i in {1,...,5} {
        \draw[thin, opacity=0.7] (0,\i) -- (4,\i);
        \draw[thin, dashed, opacity=0.7] (0,6.5) -- (4,\i);
    }

    % Key for understanding the graph
    \node at (6,5.5) [rectangle, draw] {\begin{tabular}{c l}
        \tikz\draw[fill=blue!30] (0,0) circle (0.5ex) & - Hombre \\
        \tikz\draw[fill=red!30] (0,0) circle (0.5ex) & - Mujer \\
        \tikz\draw[pattern=north east lines] (0,0) circle (0.5ex) & - Incierto \\
        \tikz\draw[thin, opacity=0.7] (0,0) -- (0.5,0) & - Conexión conocida \\
        \tikz\draw[thin, dashed, opacity=0.7] (0,0) -- (0.5,0) & - Conexión incierta \\
    \end{tabular}};
\end{tikzpicture}
\end{center}

Esta visualización indica que para cada hombre y mujer, hay conexiones directas entre ellos. Sin embargo, si el número de hombres y mujeres es el mismo, no puede sumar un total de 83 personas. Por lo tanto, la afirmación inicial es incorrecta.

\end{document}

