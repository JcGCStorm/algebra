\documentclass{article}
\usepackage{amsmath, amssymb}

\begin{document}

\title{Árboles cuyo complemento también es un árbol}
\author{ChatGPT}
\maketitle

\section{Introducción}
Caracterizamos todos los árboles tales que su complemento también es un árbol, basándonos en definiciones y propiedades básicas de árboles y grafos complementarios.

\section{Definiciones}
\begin{itemize}
    \item Un \textbf{árbol} es un grafo conexo sin ciclos.
    \item El \textbf{complemento} de un grafo $G$ es otro grafo que tiene el mismo conjunto de vértices que $G$, pero cuyas aristas son precisamente las que no están en $G$.
\end{itemize}

\section{Observaciones}
\begin{enumerate}
    \item Un árbol con $n$ vértices tiene $n - 1$ aristas.
    \item Un grafo completo con $n$ vértices tiene $\frac{n(n - 1)}{2}$ aristas.
    \item Si el complemento de un grafo $G$ es un árbol, entonces el complemento debe tener $n - 1$ aristas.
\end{enumerate}

\section{Solución}
Dado un árbol $T$ con $n$ vértices y $n - 1$ aristas, su complemento $T'$ tendrá:
\[
\frac{n(n - 1)}{2} - n + 1
\]
aristas. Si el complemento $T'$ es también un árbol, entonces debe tener $n - 1$ aristas. Esto nos lleva a la ecuación:
\[
\frac{n(n - 1)}{2} - n + 1 = n - 1
\]
Resolviendo esta ecuación, encontramos que los posibles valores de $n$ son 1 y 4.

\section{Conclusión}
Los únicos árboles cuyos complementos también son árboles son aquellos que tienen 1 o 4 vértices. Cuando tienen 4 vértices, el árbol original es un camino simple y su complemento es una estrella.

\end{document}
