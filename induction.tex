\documentclass{article}
\usepackage{amssymb}

\begin{document}

\title{My First LaTeX Document}
\author{David Rivera Morales}
\date{\today}
\maketitle

\section{1. Prove that for $a,b,c,d \in \mathbb{N}$, if $a < c$, then $\exists t \in \mathbb{N}$ such that $a + t = c$.}

\begin{proof}
Assume $a < c$. Then $\exists t \in \mathbb{N}$ such that $a + t = c$. Also $b < d$ and $\exists s \in \mathbb{N}$ such that $b + s = d$.
\item $a + t + (b + s) = c + d$.
\item $(a + b) + (t + s) = c + d$.
\item $c + d = (a + b) + (t + s)$.
\item By definition of order 
In the context of natural numbers, the term "order" typically refers to the relation of "less than or equal to". We say that a natural number $a$ is less than or equal to another natural number $b$ if $a$ can be obtained from $b$ by subtracting a non-negative integer. This relation is denoted by $a \leq b$. 
Formally, we can define the order relation on natural numbers as follows:
For any natural numbers $a$ and $b$, we say that $a$ is less than or equal to $b$, denoted by $a \leq b$, if there exists a natural number $c$ such that $a + c = b$.$=> a + b < c + d$.
\item $a + b < c + d$.

\item $a < c$ and $b < d$ $=>$ $a + b < c + d$.

\end{proof}

% necessary math concepts
\begin{enumerate}
    1. **Natural Numbers ($\mathbb{N}$)**: This is written as `$\mathbb{N}$` in LaTeX. The `amssymb` package is required for `\mathbb{}`.

2. **Inequalities**: Less than and greater than symbols can be directly used in LaTeX, i.e., `$a < c$` and `$d > b$`.

3. **Existential Quantifier ($\exists$)**: This is represented as `$\exists$` in LaTeX.

4. **Addition**: You can represent addition with the plus symbol `+`, like `$a + b$`.

5. **Associative Property of Addition**: You can use parentheses to group terms and demonstrate the property, like `$(a + b) + c = a + (b + c)$`.

6. **Order Relation in Natural Numbers**: You can use `\leq` for "less than or equal to" and `\geq` for "greater than or equal to". E.g., `$a \leq b$`.

7. **Implication ($\Rightarrow$)**: In LaTeX, you can represent this as `$\Rightarrow$`. For example, `$a < c \Rightarrow a + b < c + d$`.


\end{document}