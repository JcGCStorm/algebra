\documentclass{article}
\begin{document}

\section*{Ejercicio 1 Tarea 3}

\textbf{Figura 1}

\textbf{Parametrización de un segmento de recta que pasa por $(0,-1)$ y por $(1,0)$:}

$P_0 = (0, -1)$ \\
$\vec{Q_0} - \vec{P_0} = (1, 0) - (0, -1) = (-1, 0)$ \\
$\mathbf{r}(t) = (0, -1) + (-1, 0)t$ \\
$\mathbf{r}(t) = (0 + -1t, -1 + 0t)$

\textbf{Parametrización de un segmento de recta que pasa por $(1,0)$ y por $(2,0)$:}

$P_1 = (1, 0)$ \\
$\vec{Q_1} - \vec{P_1} = (2, 0) - (1, 0) = (1, 0)$ \\
$\mathbf{r}(t) = (1, 0) + (1, 0)t$ \\
$\mathbf{r}(t) = (1 + 1t, 0 + 0t)$

\textbf{Parametrización de un segmento de recta que pasa por $(2,0)$ y por $(0,-2)$:}

$P_2 = (2, 0)$ \\
$\vec{Q_2} - \vec{P_2} = (0, -2) - (2, 0) = (-2, -2)$ \\
$\mathbf{r}(t) = (2, 0) + (-2, -2)t$ \\
$\mathbf{r}(t) = (2 + (-2)t, 0 + (-2)t)$

\textbf{Parametrización de un segmento de recta que pasa por $(0,-2)$ y por $(0,-1)$:}

$P_3 = (0, -2)$ \\
$\vec{Q_3} - \vec{P_3} = (0, -1) - (0, -2) = (0, 1)$ \\
$\mathbf{r}(t) = (0, -2) + (0, 1)t$ \\
$\mathbf{r}(t) = (0 + 0t, -2 + 1t)$

\textbf{Figura 2}

\textbf{Segmento de recta que pasa por $(6,0,4)$ y por $(6,3,0)$:}

$P_4 = (6,0,4)$ \\
$\vec{Q_4} - \vec{P_4} = (6,3,0) - (6,0,4) = (0, -3, -4)$ \\
$\mathbf{r}(t) = (6,0,4) + (0, -3, -4)t$ \\
$\mathbf{r}(t) = (6 + 0t, 0 + (-3)t), 4 + (-4)t)$

\textbf{Segmento de recta que pasa por $(6,3,0)$ y por $(0,3,0)$:}

$P_5 = (6,3,0)$ \\
$\vec{Q_5} - \vec{P_5} = (0,3,0) - (6,3,0) = (-6,0,0)$ \\
$\mathbf{r}(t) = (6,3,0) + (-6,0,0)t$ \\
$\mathbf{r}(t) = (6 + (-6)t, 0 + 0t, 0 + 0t)$

\textbf{Segmento de recta que pasa por $(0,3,0)$ y por $(0,0,4)$:}

$P_6 = (0,3,0)$ \\
$\vec{Q_6} - \vec{P_6} = (0,0,4) - (0,3,0) = (0, -3, 4)$ \\
$\mathbf{r}(t) = (0,3,0) + (0, -3, 4)t$ \\
$\mathbf{r}(t) = (0 + 0t, 3 + (-3)t, 0 + 4t)$

\textbf{Segmento de recta que pasa por $(0,0,4)$ y por $(6,0,4)$:}

$P_7 = (0,0,4)$ \\
$\vec{Q_7} - \vec{P_7} = (6,0,4) - (0,0,4) = (6,0,0)$ \\
$\mathbf{r}(t) = (0,0,4) + (6,0,0)t$ \\
$\mathbf{r}(t) = (0 + 6t, 0 + 0t, 4 + 0t)$

\end{document}
